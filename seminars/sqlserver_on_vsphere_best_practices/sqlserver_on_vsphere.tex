\documentclass[CJK]{beamer}
\usetheme{Boadilla}
\usepackage{graphicx}
\usepackage{hyperref}

\title[MS SQL Server on vSphere Best Practices]{MS SQL Server on vSphere Best Practices}
\author{Raymond Wen}

\begin{document}

\begin{frame}
    \titlepage
\end{frame}

\begin{frame}
    \frametitle{Topics}
    \tableofcontents[pausesections]
\end{frame}

\section{vSphere Basics}
\subsection{Why Use vSphere}
\begin{frame}[t]
    \frametitle{Why Use vSphere}
        \begin{itemize}
            \item Scalability
            \item Management
            \item Availability
            \item Cost
            \item Consolidation
            \pause
            \begin{figure}[b]
                \includegraphics[width=\textwidth]{"server_consolidation.png"}
            \end{figure}
        \end{itemize}
\end{frame}

\subsection{vSphere Architecture}
\begin{frame}[t]
    \frametitle{vSphere Architecture}
    \begin{figure}[b]
        \includegraphics[width=250pt]{"vsphere_architecture.png"}
    \end{figure}
\end{frame}

\begin{frame}[t]
    \frametitle{vSphere Components}
        \begin{itemize}
            \item VMware ESXi
            \item VMware vCenter Server
            \item VMware vSphere Client/Web Client/CLI/SDK
            \item vSphere Virtual Machine File System (VMFS)
            \item vSphere Virtual SMP
            \item vSphere vMotion
            \item vSphere Storage vMotion
            \item vSphere High Availability
            \item vSphere Distributed Resource Scheduler (DRS)
            \item vSphere Storage DRS
            \item vSphere Fault Tolerance
        \end{itemize}
\end{frame}

\section{Migrating SQL Server to vSphere}
\begin{frame}[t]
    \frametitle{Migrating SQL Server to vSphere}
    \begin{itemize}
        \item Understand database workloads
        \item Understand availablity and recovery requirements
        \item Capture resource utilization baseline
        \item Plan the migration
        \item Understand database consolidation approaches
    \end{itemize}
\end{frame}

\begin{frame}[t]
    \frametitle{Understand database workloads}
    The classfication of database server indiciates resource requirements.
    \begin{itemize}
        \item Mission-critical databases (Tier 1 databases)
        \item Essential databases
        \item Support databases
    \end{itemize}
    Resource needs are defined in terms of CPU, memory, disk and network IO, user connections, transaction throughput, query effiency/latencies, database size.
\end{frame}

\begin{frame}[t]
    \frametitle{Understand availability and recovery requirements}
    Understand the database's availability and recovery requirements helps identify technology to use.
    \begin{itemize}
        \item vSphere Technologies
        \begin{itemize}
            \item vSphere High Availability
            \item vSphere Fault Tolerance
            \item vSphere DRS
            \item vSphere vMotion
        \end{itemize}
        \item \href{http://msdn.microsoft.com/en-us/library/ms190202(v=sql.110).aspx}{Microsoft SQL Server Technologies}
        \begin{itemize}
            \item \href{http://msdn.microsoft.com/en-us/library/ms189852(v=sql.110).aspx}{Database Mirroring}
            \item \href{http://msdn.microsoft.com/en-us/library/hh510230(v=sql.110).aspx}{AlwaysOn Availability Groups}
            \item \href{http://msdn.microsoft.com/en-us/library/ms189134(v=sql.110).aspx}{AlwaysOn Failover Cluster Instances}
            \item \href{http://msdn.microsoft.com/en-us/library/ms187103(v=sql.110).aspx}{Log Shipping}
        \end{itemize}
        \item Backup and Restore
            \begin{itemize}
                \item In-guest, software agent-based backup
                \item VMware Data Recovery
                \item Array-based backup
            \end{itemize}
    \end{itemize}
\end{frame}

\begin{frame}[t]
    \frametitle{Capture resource utilization baseline}
    \begin{itemize}
        \item Establish the baseline on physical deployment
        \item VMware Capacity Planer
        \item Use performance counter
        \item Collect the data over a period of time long enough to reflect all variations in the usage patterns
    \end{itemize}
\end{frame}

\begin{frame}[t]
    \frametitle{Plan the migration}
    \begin{itemize}
        \item Consider OS, SQL Server version, patches, hotfix needs
        \item Design vSphere architeture meets requirements
        \item Follow \href{http://technet.microsoft.com/en-us/sqlserver/jj901640}{Microsoft Guidances}
    \end{itemize}
\end{frame}

\begin{frame}[t]
    \frametitle{Understand database consolidation approaches - Tranditional}
        \includegraphics[width=\textwidth]{"sqlserver_consolidation_tran.png"}
\end{frame}

\begin{frame}[t]
    \frametitle{Understand database consolidation approaches - vSphere}
        \includegraphics[width=\textwidth]{"sqlserver_consolidation_vsphere.png"}
\end{frame}

\section{vSphere Best Practices}
\subsection{vSphere Host Best Practices for SQL Server}
\begin{frame}[t]
    \frametitle{vSphere Host Bset Practices for SQL Server}
    \begin{itemize}
        \item Use vSphere which performs better than hosted products, VMware Workstation, VMware Fusion
        \item Use the latest hardware 
        \item Follow \href{http://technet.microsoft.com/en-us/sqlserver/bb671430.aspx}{Microsoft SQL Server best practices}
    \end{itemize}
\end{frame}

\begin{frame}[t]
    \frametitle{vSphere Host Bset Practices for SQL Server}
    CPU configuration
    \begin{itemize}
        \item CPU Allocation
        \begin{itemize}
            \item Maintain 1:1 ratio of vCPU and CPU for tier 1 database for lower latency
            \item Reasonable CPU overcommitment for lower tier 1 database for larger throughput
        \end{itemize}
        \item NUMA Considerations
        \begin{itemize}
            \item Allocate vCPU number less than the number of cores in each physical NUMA node
            \item Configure vNUMA to enable SQL Server NUMA optimization to take advantage of managing memory locality
        \end{itemize}
        \item Hyperthreading
        \begin{itemize}
            \item Disable hyperthreading for tier 1 databases
            \item Enable hyperthreading for higher throughput
        \end{itemize}
    \end{itemize}
\end{frame}

\begin{frame}[t]
    \frametitle{vSphere Host Bset Practices for SQL Server}
    Memory configuration
    \begin{itemize}
        \item Set virtual machine memory size to less than NUMA node memory amount to avoid latency
        \item Enable Large Page
        \item Set memory reservation for tier 1 database to avoid memory ballooning and swapping
    \end{itemize}
\end{frame}

\begin{frame}[t]
    \frametitle{vSphere Host Bset Practices for SQL Server}
    Storage configuration
    \begin{itemize}
        \item VMFS vs RDM
        \item In terms of storage protocol, Fibre Channel performs better than iSCSI and NFS
        \begin{figure}[b]
            \includegraphics[width=200pt]{"storage_protocol_compare.png"}
        \end{figure}
        \item Avoid lazy zeroing
        \item Place SQL Server binary, log and data files to separate VMDKs
    \end{itemize}
\end{frame}

\begin{frame}[t]
    \frametitle{vSphere Host Bset Practices for SQL Server}
    Network configuration
    \begin{itemize}
        \item Use NIC teaming
        \item Use separate physical NICs for management traffic and virtual machine traffic
        \item Use dedicate network adapter for iSCSI
    \end{itemize}
\end{frame}

\subsection{SQL Server In-Guest Best Practices}
\begin{frame}[t]
    \frametitle{vSphere In-Guest Best Practices}
    \begin{itemize}
        \item Max and min memory configuration
        \item Grant lock pages priviledge to database account
        \item Use large pages
    \end{itemize}
\end{frame}

\section{Performance Metrics}
\subsection{Performance Metrics}
\begin{frame}[t]
    \frametitle{Performance Metrics}
    Test environment
    \begin{itemize}
        \item High-end OLTP workload derived from \href{http://www.tpc.org/tpce/default.asp}{TPC-E1}
        \item 2.69G AMD Opteron, 64GB memory
        \item Windows 2008, SQL Server 2008 Enterprise, VMware ESX 4.0
    \end{itemize}
    \begin{figure}[b]
        \includegraphics[width=180pt]{"test_environment_hw.png"}
    \end{figure}
    \url{http://www.vmware.com/files/pdf/perf_vsphere_sql_scalability.pdf}
\end{frame}

\begin{frame}
    \frametitle{Performance Metrics}
    \begin{figure}[b]
        \includegraphics[width=\textwidth]{"scale_up_performance_vs_native.png"}
    \end{figure}
\end{frame}

\begin{frame}
    \frametitle{Performance Metrics}
    \begin{figure}[b]
        \includegraphics[width=\textwidth]{"scale_out_consolidation_performance.png"}
    \end{figure}
\end{frame}

\subsection{Ongoing Performance Monitoring and Tuning}
\begin{frame}[t]
    \frametitle{Ongoing Performance Monitoring and Tuning}
    \begin{itemize}
        \item vSphere client
        \item resxtop
        \item SQL Server monitoring tools: Performance Counters, Profiler, Dynamic Management Views
        \item Focus on performance bottlenecks instead of time-base measurements
    \end{itemize}
\end{frame}

\end{document}
